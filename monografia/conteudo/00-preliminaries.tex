%% ------------------------------------------------------------------------- %%
\chapter{Preliminaries}
\label{cap:introducao}

\section{Introduction}

Reinforcement Learning is considered a subfield of Machine Learning, where learning occurs through an agent interacting with an environment. At each time step, the agent performs an action and the environment responds by producing a reward signal and transitioning to the next state. The goal of the agent is to maximize the total expected reward. \cite{suttonbarto}

There are a lot of challenges that naturally arise from reinforcement learning problems that differ from the ones faced in classic machine learning. For example, balancing immediate rewards and future rewards: up to which point is it worth to sacrifice early rewards in exchange for bigger rewards in the future? An agent might be inclined to take actions that have been taken before because it has the knowledge of how much reward those actions will yield. Thus, limiting the agent to that specific set of actions and ultimately impairing it from further exploring the environment and possibly discovering new states and actions that could yield even more rewards. This is called the exploration-explotation dillema. \cite{suttonbarto}


The focus of this work will be an application of the Proximal Policy Optimization (PPO) algorithm to train an agent able to park a car in a designated spot.

\section{Motivation}
With car crashes being more and more common, car manufacturers have started working on technologies to avoid crashes, ranging from simple proximity sensors that warns the driver of a imminent collision to fully fledged auto-driving systems. In the latter, automated parking is a key part in autonomous vehicle systems that allows cars to navigate through a parking lot completely unassisted.

We aim to recreate the self-parking system inside a 3D virtual environment using deep reinforcement learning and studying how the algorithm performs in different parking situations.